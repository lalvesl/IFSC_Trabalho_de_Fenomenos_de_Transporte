{\let\clearpage\relax\chapter{Introdução}\label{cap:definitions}}
%\chapter{Introdução}\label{cap:introducao}

\vspace{-1.5cm}Este relatório tem como objetivo principal apresentar uma análise numérica dos fenômenos de transferência de calor em aletas. Estas superfícies são consideradas como uma extensão de uma base condutora, potencializando o efeito da dissipação de calor.

O estudo se concentra em aletas planas unidimensionais e tem como objetivo determinar os principais fatores que influenciam a transferência de calor nesses tipos de superfícies. A análise será feita por meio de simulações numéricas que permitirão quantificar a transferência de calor e avaliar a eficiência das aletas em dissipar o calor de forma adequada.

\section{Objetivos}

Dentro do escopo do objetivo principal, foram definidos alguns objetivos específicos:

\vspace{-1cm}\begin{itemize}[leftmargin=2cm]
   \item Determinar a temperatura da superfície da base da aleta (Tb);

   \item Determinar a taxa de transferência de calor de cada tipo de aleta (qa);

   \item Determinar a temperatura na ponta da aleta;

   \item Determinar a efetividade ({\large \(\epsilon\)}) da superfície estendida. Caso a aleta apresente ({\large\(\epsilon\)}1>2);

   \item Determinar a eficiência da aleta ({\large\(\eta\)}a);

   \item Determinar a eficiência global do conjunto aletado ({\large\({\eta}\)}o).
\end{itemize}
