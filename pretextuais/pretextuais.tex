%-----------------------------------------------%
% Folha de rosto
% (o * indica que haverá a ficha bibliográfica)
%-----------------------------------------------%
% \imprimirfolhaderosto*
\imprimirfolhaderosto
%-----------------------------------------------%

%-----------------------------------------------%
% ficha bibliográfica
% 
% Pegue com a Biblioteca do IFSC um PDF com a 
% ficha correta, salve o arquivo no diretório
% deste projeto e descomente as linhas abaixo
% \begin{fichacatalografica}
%     \includepdf{ficha-catalografica.pdf}
% \end{fichacatalografica}
%-----------------------------------------------%

%-----------------------------------------------%


%-----------------------------------------------%
% folha de aprovação
%-----------------------------------------------%
%\begin{folhadeaprovacao}
%
%    \begin{center}
%        {\ABNTEXchapterfont\large\imprimirautor}
%
%        \vspace*{\fill}\vspace*{\fill}
%        \begin{center}
%            \ABNTEXchapterfont\Large\imprimirtitulo
%        \end{center}
%        \vspace*{\fill}
%
%        \imprimirtextoaprovacao
%
%        \vspace*{1cm}
%
%        \imprimirlocal, 10 de abril de 2023:
%
%        \vspace*{\fill}
%    \end{center}
%
%    % Alterando o espaço para assinatura de 0.7cm para 1.5cm
%    \setlength{\ABNTEXsignskip}{1.5cm}
%
%    \assinatura{\textbf{\imprimirorientador} \\ Orientador\\Instituto Federal de Santa Catarina}     
%    \assinatura{\textbf{Professor Fulano, Dr.} \\ Instituto Federal de Santa Catarina }
%    \assinatura{\textbf{Professora Fulana, Dra. } \\ Instituto Federal de Santa Catarina}
%    % \assinatura{\textbf{Professor Beltrano, Dr.} \\ Instituto Z}
%
%    \vspace*{1cm}
%  
%\end{folhadeaprovacao}
%-----------------------------------------------%


%-----------------------------------------------%
% Dedicatória
%-----------------------------------------------%
%\begin{dedicatoria}
%    \vspace*{\fill}
%    \begin{flushright}
%    \noindent
%    \textit{ Este trabalho é dedicado às crianças adultas que,\\
%    quando pequenas, sonharam em se tornar cientistas.}\vspace*{2cm}
%    \end{flushright}
% \end{dedicatoria}
%
%-----------------------------------------------%


%-----------------------------------------------%
% Agradecimentos
%-----------------------------------------------%
%\begin{agradecimentos}
%    Os agradecimentos principais são direcionados à Gerald Weber, Miguel Frasson,
%    Leslie H. Watter, Bruno Parente Lima, Flávio de Vasconcellos Corrêa, Otavio Real
%    Salvador, Renato Machnievscz\footnote{Os nomes dos integrantes do primeiro
%    projeto abn\TeX\ foram extraídos de
%    \url{http://codigolivre.org.br/projects/abntex/}} e todos aqueles que
%    contribuíram para que a produção de trabalhos acadêmicos conforme
%    as normas ABNT com \LaTeX\ fosse possível.
%    
%    Agradecimentos especiais são direcionados ao Centro de Pesquisa em Arquitetura
%    da Informação\footnote{\url{http://www.cpai.unb.br/}} da Universidade de
%    Brasília (CPAI), ao grupo de usuários
%    \emph{latex-br}\footnote{\url{http://groups.google.com/group/latex-br}} e aos
%    novos voluntários do grupo
%    \emph{\abnTeX}\footnote{\url{http://groups.google.com/group/abntex2} e
%    \url{http://www.abntex.net.br/}}~que contribuíram e que ainda
%    contribuirão para a evolução do \abnTeX.
%\end{agradecimentos}
%-----------------------------------------------%


%-----------------------------------------------%
% Epígrafe
%-----------------------------------------------%
%\begin{epigrafe}
%    \vspace*{\fill}
%    \begin{flushright}
%        \textit{Sempre que te perguntarem se podes fazer um trabalho,\\
%        respondas que sim e te ponhas em seguida a aprender como se faz.\\
%        T. Roosevelt}
%    \end{flushright}
%\end{epigrafe}
%-----------------------------------------------%


%-----------------------------------------------%
% Resumo e abstract
%-----------------------------------------------%
% ajusta o espaçamento dos parágrafos do resumo
%\setlength{\absparsep}{18pt} 
%\begin{resumo}
%    O resumo deve ressaltar o objetivo, o método, os resultados e as conclusões do documento. A ordem e a extensão destes itens dependem do tipo de resumo (informativo ou indicativo) e do tratamento que cada item recebe no documento original. O resumo deve ser precedido da referência do documento, com exceção do resumo inserido no próprio documento. O resumo deve conter apenas um parágrafo com no mínimo 150 e no máximo 250 palavras. As palavras-chave devem figurar logo abaixo do resumo, antecedidas da expressão Palavras-chave:, separadas entre si por ponto e finalizadas também por ponto. Este documento segue as normas da \gls{ABNT} e para isso faz uso do pacote \gls{abnTeX}.
%    
%    \textbf{Palavras-chave}: latex. abntex. editoração de texto.
%\end{resumo}

%-----------------------------------------------%
%\begin{resumo}[Abstract]
%\begin{otherlanguage*}{english}
%    This is the english abstract.
%\vspace{\onelineskip}
%
%\noindent 
%\textbf{Keywords}: latex. abntex. text editoration.
%\end{otherlanguage*}
%\end{resumo}
%-----------------------------------------------%